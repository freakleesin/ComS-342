\documentclass[11pt]{article}

\usepackage{fullpage,times,color}%charter}
\definecolor{timberwolf}{rgb}{0.86, 0.84, 0.82}

\definecolor{armygreen}{rgb}{0.00, 0.5, 0.0}
\usepackage{xcolor}
\usepackage[disable]{todonotes}
%\usepackage[draft]{todonotes}

\newcommand{\comment}[1]{}
%\newcommand{\comment}[1]{\par {\bfseries \color{blue} #1 \par}} 
%\renewcommand{\comment}[1]{\par {\color{blue} #1} \par}
\newcommand{\betar}[1]{\stackrel{\beta-\mbox{\small reduce on}\ #1}{\longrightarrow}}
\newcommand{\alphar}{\stackrel{\beta-\mbox{\small rename}}{\longrightarrow}}


\newcommand{\lmda}[2]{\lambda #1.#2}
\newcommand{\lmd}[3]{(\lmda{#1}{#2}\ #3)}
\newcommand{\appl}[2]{(#1\ \ #2)}


\begin{document}

\begin{center}
{\Large\bf Homework: Lambda Calculus}

\medskip

{Due-date: Feb 13 at 11:59pm.}
\\
Submit online on Canvas (\textbf{\color{red} Format: pdf}).
\end{center}

\hrule

\textit{
\noindent
\\ Homework must be individual's original work. Collaborations and
discussions of any form with any students or other faculty members are
not allowed. If you have any questions and/or concerns, post them on
Piazza and/or ask 342 instructor and TAs.\\ 
} \hrule
\medskip


\noindent
\colorbox{timberwolf}{Learning Outcomes}
\begin{itemize}
\item Application of knowledge of computing and mathematics
\item Ability to understand the implications of mathematical formalisms
      in computer science
\end{itemize}

\medskip

\noindent
\colorbox{timberwolf}{Questions}
\begin{enumerate}
\item Compute the following (apply reductions till the
expression cannot be reduced any further)
\begin{enumerate}

\item 
$\appl{\appl{\lmda{x}{\appl{x}{x}}}{\lmda{y}{y}}}{\lmda{y}{y}}$

\item 
$\appl{\appl{\lmda{x}{\lmda{y}{\appl{x}{\appl{y}{y}}}}}{\lmda{a}{a}}}{b}$

\item 
$\appl{\appl{\lmda{x}{\appl{x}{x}}}{\lmda{y}{\appl{y}{x}}}}{z}$

\item
$\appl{\lmda{g}{\appl{g}{\lmda{x}{\lmda{y}{x}}}}}{\appl{\appl{\lmda{a}{\lmda{b}{\lmda{h}{\appl{\appl{h}{a}}{b}}}}}{z_1}}{z_2}}$

\item
$\appl{\appl{\lmda{t}{\lmda{y}{\appl{t}{y}}}}{\lmda{n}{\lmda{f}{\lmda{x}{\appl{f}{\appl{\appl{n}{f}}{x}}}}}}}{\lmda{g}{\lmda{z}{\appl{g}{\appl{g}{z}}}}}$

\end{enumerate}
\hfill (15pts)

\item 
Given the following lambda expressions and corresponding
interpretations:
\begin{itemize}
\item The interpretation of $\lmda{f}{\lmda{x}{x}}$ is natural number
  $0$ (zero). The interpretation of
  $\lambda{f}{\lmda{x}{\appl{f}{\appl{f}{\appl{...}{x}}}}}$, with $n$
  applications of $f$ on $x$, is the natural number $n>0$.
\item The interpretation of
  $\lmda{n}{\lmda{f}{\lmda{x}{\appl{f}{\appl{\appl{n}{f}}{x}}}}}$ is a
  successor function $succ$ for natural numbers, where $n$ is the
  formal parameter corresponding to the number whose successor is
  computed.
\item The interpretation of
  $\lmda{m}{\lmda{n}{\appl{\appl{m}{succ}}{n}}}$
  is the addition function $add$ for two natural numbers, where $m$
  and $n$ are the formal parameters corresponding to the numbers whose
  sum is computed.
\item The interpretation of 
 $\lmda{m}{\lmda{n}{\appl{\appl{m}{\appl{add}{n}}}{zero}}}$
is the multiplication function $mul$ for two natural numbers, where $m$
  and $n$ are the formal parameters corresponding to the numbers whose
  product is computed.
\item The interpretation of $\lmda{x}{\lmda{y}{x}}$ is propositional constant $true$.
\item The interpretation of $\lmda{x}{\lmda{y}{y}}$ is propositional constant $false$.

\item The interpretation of
  $\lmda{a}{\lmda{b}{\lmda{h}{\appl{\appl{h}{a}}{b}}}}$ is
  a \emph{pair} of entities $a$ and $b$ on which some function $h$ can
  be applied. We will refer to this function as $Pair$. The first or
  second element of the pair $\appl{\appl{Pair}{z_1}}{z_2}$ can be
  obtained by applying on it the functions
  $\lmda{g}{\appl{g}{\lmda{a}{\lmda{b}{a}}}}$ (referred to as $fst$)
  and $\lmda{g}{\appl{g}{\lmda{a}{\lmda{b}{b}}}}$ (referred to as
  $sec$), respectively. That is.
  $\appl{fst}{\appl{\appl{Pair}{z_1}}{z_2}} = z_1$ and
  $\appl{sec}{\appl{\appl{Pair}{z_1}}{z_2}} = z_2$.
%
\item The interpretation of a pair $\appl{\appl{Pair}{m}}{n}$ where
  $m$ and $n$ are natural numbers is a signed number whose valuation
  is difference between $m$ and $n$ (i.e., $m-n$). For instance,\linebreak
  $\appl{\appl{Pair}{\lmda{f}{\lmda{x}{x}}}}{\lmda{f}{\lmda{x}{\appl{f}{x}}}}$
  represents a signed number $-1$.

\end{itemize}
Identify the mathematical/logical interpretation for the following
expressions. Justify your answer. (In all these problems, apply the
functions on some actual arguments and examine the results; does the
result correspond to some interpretation that you already know
about---basic arithmetic or logical operations. We have done similar
problems, when we identified the interpretation of functions
representing addition and multiplication of naturals, and negation,
conjunction and disjuction of propositions.).
%
\begin{enumerate}
\item 
$\lmda{x}{\appl{\appl{x}{false}}{true}}$, where $x$ is the formal
parameter corresponding to \textbf{propositional constants}.

\item 
$\lmda{n}{\appl{\appl{n}{\lmda{p}{\appl{\appl{p}{false}}{true}}}}{false}}$,
where $n$ is the formal parameter corresponding to \textbf{natural
  numbers}. 

\item 
$\lmda{m}{\lmda{n}{\appl{\appl{m}{\appl{mul}{n}}}{\appl{succ}{zero}}}}$,
  where $m$ and $n$ are formal parameters corresponding to \textbf{natural
  numbers}.

\item $\lmda{p}{\appl{\appl{Pair}{\appl{sec}{p}}}{\appl{fst}{p}}}$,
where $p$ is the formal parameter correspond to some \textbf{signed number}.

\item 
$\lmda{p_1}{\lmda{p_2}{\appl{\appl{Pair}{\appl{\appl{add}{\appl{fst}{p_1}}}{\appl{sec}{p_2}}}}{\appl
{\appl{add}{\appl{sec}{p_1}}}{\appl{fst}{p_2}}}}}$, where $p_1$ and
$p_2$ are formal paratemeters corresponding to some \textbf{signed
numbers}.

\end{enumerate}
\hfill(25pts)

\end{enumerate}

\hrule

\smallskip
\noindent
\emph{Notes: For any lambda expression, whenever you see an application
of the form $\appl{e_1}{e_2}$, where $e_1$ is a lambda abstraction and
$e_2$ is a ``large'' expression, use some variable (e.g., $\varphi$)
to represent the large expression. Perform the $\beta$-reduction in
the application then expand the expression $e_2$, if necessary.
Carefully consider the``($\ldots$)''-matching.}

\begin{center}
\textbf{Due date: Feb 13 at 11:59pm.}
\end{center}
\smallskip

\hrule

\end{document}
